%!TEX root = touchdevelop_tutorials.tex
\section{Introduction}
% no \IEEEPARstart

Programming has, over the last 60 years, become an increasingly popular discipline, both as a primary profession and as a hobby or additional job skill. Traditionally most self-taught programmers learn from a book or a website that intersperses writing with code snippets that can be copied and pasted into a source code editor.
However, in recent years the advent of rich web apps that are able to respond to the user dynamically have begun to change how programming is taught. The website Codecademy [4], for example, presents all its tutorials via a web app that also functions as an IDE, complete with progress indicators, hints, and tests to make sure the user�s code is correct. The site Code School [5] advertises by saying �Enjoy an education in the comfort of your browser�. Another new source of interactive programming tutorials is TouchDevelop [9]. This platform is intended to teach and enable end-user programmers on touch-based devices such as smartphones or tablets to create their own scripts. As both a new programming paradigm and a new programming language, it necessarily involves a large amount of teaching users new to programming, TouchDevelop, or both.

\begin{itemize}
\item What effect do tutorials have on end-user programmer feature usage?
\item What effect do tutorials have on end-user programmer engagement?
\end{itemize}