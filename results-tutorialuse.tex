%!TEX root = touchdevelop_tutorials.tex

\subsection{\rqtutorial}

In general, TouchDevelop script authors are only somewhat willing to attempt tutorials. Of the 18022 users who have published at least one script, we observed that only 2981 (16.5\%) attempted a tutorial and 2513 (14\%) of all users completed one or more tutorials.

\note{Implication?} Despite this finding, our data still shows that a large number of users have successfully published scripts. This suggests that many TouchDevelop users are either familiar with programming and did not require tutorials or that they learned or were taught TouchDevelop offline (for example, as part of an outreach effort or by friends). There is also the possibility that users generally avoided the official tutorials and used other resources, but non-Microsoft tutorials are rather rare. \note{One way to explore this further is to examine the features of scripts that people who didn't use tutorials had. If they are filled with advanced constructs then we can assume that they have some prior knowledge}

\begin{table}
	\centering
	\begin{tabular}{rr}
		\toprule
		Authors who published one or more scripts & 18015 (100\%) \\
		\midrule
		Authors who did not attempt a tutorial & 15045 (83.5\%) \\
		Authors who attempted or completed a tutorial & 2970 (16.5\%)\\
		\midrule
		Authors who attempted (but did not complete) a tutorial & 468 (2.6\%) \\
		Authors who completed one tutorial & 1977 (11.0\%) \\
		Authors who completed more than tutorial & 525 (3.0\%) \\
		\bottomrule
	\end{tabular}
	\caption{Proportion of TouchDevelop users who complete tutorials. Only 16.5\% of users attempt tutorials and 14\% complete at least one tutorial.}
	\label{tab:tutorial_completions}
\end{table}




\note{The correlations are...} The correlations appear in Table \ref{tab:correlations_author}. The correlations are really strong for a lot of stuff and this needs to be interpreted. Many of these make sense, as one would expect that the more scripts someone has, the more features they'll use and the more subscribers/score they'll get.

% latex table generated in R 3.0.1 by xtable 1.7-1 package
% Fri Jan 17 13:00:05 2014
\begin{table}[ht]
\centering
\begin{tabular}{rllllll}
  \hline
 & Scripts & Tutorials completed & Features & Days active & Positive reviews & Followers \\ 
  \hline
number\_of\_scripts &  &  &  &  &  &  \\ 
  number\_of\_completed\_tutorials &  0.50*** &  &  &  &  &  \\ 
  features &  0.26*** &  0.15*** &  &  &  &  \\ 
  activedays &  0.31*** &  0.08*  &  0.74*** &  &  &  \\ 
  receivedpositivereviews &  0.09**  &  0.05  &  0.52*** &  0.60*** &  &  \\ 
  subscribers &  0.07*  &  0.11*** &  0.34*** &  0.43*** &  0.77*** &  \\ 
  score &  0.28*** &  0.10**  &  0.91*** &  0.90*** &  0.70*** &  0.52*** \\ 
   \hline
\end{tabular}
\label{tab:correlations_author}
\caption{Correlations between counts of items per author \note{this needs a better caption}}
\end{table}







