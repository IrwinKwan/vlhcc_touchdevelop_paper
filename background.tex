%!TEX root = touchdevelop_tutorials.tex
\section{Background}

\subsection{TouchDevelop}

TouchDevelop is an experimental programming platform for end-user programming on touch-screen devices. It is also designed a research tool, meaning that published projects are automatically open sourced, and a wealth of data on users and published scripts is made public (see [3]) on a growing userbase. However, while the JSON-based APIs are very useful for looking at specific scripts or examining scripts one at a time, large-scale relational-database style queries are unavailable in the TouchDevelop API.

As a result, studies of the TouchDevelop population such as Athreya et al. [1] have had to rely on a random sample of scripts, classified by hand. First and most importantly, this greatly limits the size of the sample. Secondly, it introduces personal bias and error into the process - Athreya et al.’s study [1] describes the process of ‘negotiation’ when researchers inevitably disagreed on classification of scripts.

One particular case of this was the category of No Meaningful Functionality scripts in [Balaji’s paper]. This made up a significant portion of the TouchDevelop script base, and included tutorials, among other published scripts that would not be useful to anyone besides the author. However, due undoubtedly to a desire to not waste time, this category of scripts was not detailed further.

Li et al. [2] studied the behavior of users in TouchDevelop. They found that 68.3\% of TouchDevelop users published one or two scripts and then never returned or produced more content. This begs the question: why? This is an especially important question, as TouchDevelop is a novel approach to programming and is intended for end user programmers, many of whom may have had no previous experience programming at all, and thus do not even know basic concepts such as variables and functions.

\subsection{Tutorials and Community Engagement}

\boldnote{Programming languages are difficult to learn, and many resources have been devoted to increasing retention and interest in end-user programming languages.} A new programming language is difficult to learn, especially for an end-user developer. These end-user developers often need to learn on their own, without the support of peers.

Analysis of end-user developer repositories
\cite{athreya2012:touchdevelop,bogart2008:coscripter,sihan2013:touchdevelop,stolee2013:yahoopipes} has shown that reuse between scripts is very low. In addition, user retention is also very low. In what ways can we improve user retention? 

However, little research has been done specifically investigating what effect making training material, such as tutorials, available to end-user programmers has on their engagement and adoption of the platform.

\note{Need argument here that we should understand effects of tutorials on user engagement}
Open-source projects have their own ways of maintaining documentation~\cite{dagenais2010} (cite other papers about the use of documentation for learning APIs, libraries, and programming). However, Carroll and Rosson~\cite{carroll_paradox_1987} identified that tutorials were also not appropriate for ``active users'', who were more concerned about completing tasks than learning.

\boldnote{The TouchDevelop environment had interactive tutorials and non-interactive tutorials and they were different in a few ways}
The TouchDevelop environment has interactive tutorials and non-interactive tutorials. From a learning standpoint, interactive tutorials seem superior; for example ... Stencils (Kelleher), LemonAid (Andy Ko). There are reasons to consider non-interactive tutorials. For instance, a study of \emph{opportunistic programmers} identified them as quickly copying and pasting snippets from tutorials to learn new skills and approaches, which would not be possible with an interactive tutorial~\cite{brandt2009:opportunistic}.

\note{Strangely enough, the tutorials don't have any accompanying plain English language text with them. Some of them have documentation, some do not.}


There are two types of TouchDevelop tutorials available: interactive and non-interactive. Interactive tutorials engage the user into tutorial completion more actively than non-interactive tutorials by providing frequent direct suggestions on what user is supposed to do next. On the other hand, non-interactive tutorials provide tutorial instructions and source code snippets without guiding the user directly.
